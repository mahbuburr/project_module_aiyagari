% Copyright 2004 by Till Tantau <tantau@users.sourceforge.net>.
%
% In principle, this file can be redistributed and/or modified under
% the terms of the GNU Public License, version 2.
%
% However, this file is supposed to be a template to be modified
% for your own needs. For this reason, if you use this file as a
% template and not specifically distribute it as part of a another
% package/program, I grant the extra permission to freely copy and
% modify this file as you see fit and even to delete this copyright
% notice. 

\documentclass{beamer}
\usepackage{graphicx}
\usepackage{breqn}
\graphicspath{ {images/} }
\usepackage{amsmath}
\usepackage{dsfont}
% There are many different themes available for Beamer. A comprehensive
% list with examples is given here:
% http://deic.uab.es/~iblanes/beamer_gallery/index_by_theme.html
% You can uncomment the themes below if you would like to use a different
% one:
%\usetheme{AnnArbor}
%\usetheme{Antibes}
%\usetheme{Bergen}
%\usetheme{Berkeley}
\usetheme{Berlin}
%\usetheme{Boadilla}
%\usetheme{boxes}
%\usetheme{CambridgeUS}
%\usetheme{Copenhagen}
%\usetheme{Darmstadt}
%\usetheme{default}
%\usetheme{Frankfurt}
%\usetheme{Goettingen}
%\usetheme{Hannover}
%\usetheme{Ilmenau}
%\usetheme{JuanLesPins}
%\usetheme{Luebeck}
%\usetheme{Madrid}
%\usetheme{Malmoe}
%\usetheme{Marburg}
%\usetheme{Montpellier}
%\usetheme{PaloAlto}
%\usetheme{Pittsburgh}
%\usetheme{Rochester}
%\usetheme{Singapore}
%\usetheme{Szeged}
%\usetheme{Warsaw}

\title{Effects of Unemployment Benefits and Uncertainty in Heterogeneous Models}

% A subtitle is optional and this may be deleted
\subtitle{}

\author{Eric, Yannic and Andrian}
% - Give the names in the same order as the appear in the paper.
% - Use the \inst{?} command only if the authors have different
%   affiliation.

%\institute[Universities of Somewhere and Elsewhere] % (optional, but mostly needed)
%{
% \inst{1}%
%  Department of Computer Science\\
%  University of Somewhere
%  \and
%  \inst{2}%
%  Department of Theoretical Philosophy\\
%  University of Elsewhere}
% - Use the \inst command only if there are several affiliations.
% - Keep it simple, no one is interested in your street address.

%\date{Conference Name, 2013}
% - Either use conference name or its abbreviation.
% - Not really informative to the audience, more for people (including
%   yourself) who are reading the slides online

%\subject{Theoretical Computer Science}
% This is only inserted into the PDF information catalog. Can be left
% out. 

% If you have a file called "university-logo-filename.xxx", where xxx
% is a graphic format that can be processed by latex or pdflatex,
% resp., then you can add a logo as follows:

% \pgfdeclareimage[height=0.5cm]{university-logo}{university-logo-filename}
% \logo{\pgfuseimage{university-logo}}

% Delete this, if you do not want the table of contents to pop up at
% the beginning of each subsection:
\AtBeginSubsection[]
{
  \begin{frame}<beamer>{Outline}
    \tableofcontents[currentsection,currentsubsection]
  \end{frame}
}


% Let's get started
\begin{document}



\begin{frame}
  \titlepage
\end{frame}

\begin{frame}{Outline}
  \tableofcontents
  % You might wish to add the option [pausesections]
\end{frame}

% Section and subsections will appear in the presentation overview
% and table of contents.
\section{Part I}
\subsection{}
\begin{frame}{Motivation}
  \begin{itemize}

  \item {
  Item 1
  }

  \item {
  Item 2  
  }

  \item {
  Item 3
  }
  \end{itemize}

  \centering{\includegraphics[scale=0.30]{eq18}}
\end{frame}

\section{Part II}
\subsection{}

\section{Part III}
\subsection{}

\begin{frame}{Adding aggregate uncertainty}
  \begin{itemize}

  \item {
  Aggregate uncertainty! Why?
  }
  \pause
  \item {
  Two aggregate productivity : $z \in \{ 0.99, 1.01 \}$.
  }

  \item {
  Two unemployment states: $L \in \{ 0.1, 0.4 \}$.
  }

  \item {
  2 transition matricies between employed and unemployed respectively in good and bad states and 1 transition matrix between good and bad state: $L \in \{ 0.1, 0.4 \}$.
  }

  \pause

  \item {
  Let's solve it!.
  }

  

  \end{itemize}

\end{frame}

\begin{frame}{Problem}
  \begin{itemize}

  \item {
  The introduction of aggregate uncertainty makes the distribution of wealth changing over time, which is infinite dimensional object.
  }
  
  \item {
  The capital distribution must be taken into account by the agents, therefore it becomes state variable.
  }
 

  \end{itemize}

\end{frame}

\begin{frame}{Solution}
  \begin{itemize}

  \item {
  There is no solution.
  }
  
  \item {
  Krusell and Smith (1997) suggest an algorithm to solve the issue by an approximation, which could be summerized as "rational adaptive expectations".
  }

  \item {
  They assume that the agents forecast prices by using the future mean of the wealth distribution.
  }

  \item {
  The actual forecast is made for the future mean based on previous realizations using some Least Squares technique.
  }
 

  \end{itemize}

\end{frame}

\begin{frame}{Algorithm}
  \begin{enumerate}

  \item {
  Draw shocks.
  }

  \item {
  Guess initial coefficients for the forcasting rule.  
  }
  
  \item {
  Solve the individual problem.
  }

  \item {
  Use the policy function to generate a sequence of capital distributions.
  }

  \item {
  The actual forecast is made for the future mean based on previous realizations using some Least Squares technique.
  }

  \item {
  Estimate new coefficients based on past realizations of the distribution and iterate until they converge. 
  }
 

  \end{enumerate}

\end{frame}


\end{document}